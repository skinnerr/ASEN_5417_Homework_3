\documentclass[12pt]{article}

\input{../../Latex_Common/skinnerr_latex_preamble_asen5417.tex}

%%
%% DOCUMENT START
%%

\begin{document}

\pagestyle{fancyplain}
\lhead{}
\chead{}
\rhead{}
\lfoot{\hrule ASEN 5417: Homework 3}
\cfoot{\hrule \thepage}
\rfoot{\hrule Ryan Skinner}

\noindent
{\Large Homework 3}
\hfill
{\large Ryan Skinner}
\\[0.5ex]
{\large ASEN 5417: Numerical Methods}
\hfill
{\large Due 2015/09/29}\\
\hrule
\vspace{6pt}

%%%%%%%%%%%%%%%%%%%%%%%%%%%%%%%%%%%%%%%%%%%%%%%%%
%%%%%%%%%%%%%%%%%%%%%%%%%%%%%%%%%%%%%%%%%%%%%%%%%
\section{Introduction} %%%%%%%%%%%%%%%%%%%%%%%%%%
%%%%%%%%%%%%%%%%%%%%%%%%%%%%%%%%%%%%%%%%%%%%%%%%%
%%%%%%%%%%%%%%%%%%%%%%%%%%%%%%%%%%%%%%%%%%%%%%%%%

We solve the following problems to better understand numerical techniques for solving boundary value problems governed by ordinary differential equations. As will be described in the methods section, our tools primarily consist of simple iteration, the fourth-order Runge-Kutta method, and the secant method.

\subsection{Problem 1}

Solve the system of equations,
\begin{equation}
\begin{alignedat}{2}
x_1 + x_2 - \sqrt{x_2} - \tfrac{1}{4} &= 0 \\
8 x_1^2 - 8 x_1 x_2 + 16 x_2 - 5      &= 0
\;,
\end{alignedat}
\label{eq:prob1}
\end{equation}
by simple iteration, starting with $x_{1_0} = x_{2_0} = 0$, and with an iteration tolerance of $\epsilon = 10^{-6}$.

\subsection{Problem 2}

Calculate the boundary value problem of free convection along a vertical plate. This problem is governed by similarity equations of the form
\begin{equation}
\begin{aligned}
F''' + 3 F F'' - 2F'^2 + \theta &= 0 \\
\theta'' + 3 \text{Pr} F \theta' &= 0
\;,
\end{aligned}
\end{equation}
where $\theta = \theta(\eta)$, $F = F(\eta)$. The boundary conditions are
\begin{equation}
\begin{aligned}
\eta = 0 &: &\quad F = F' =\; &0, &\quad \theta =\; &1 \\
\eta \rightarrow \infty &: &\quad F' \rightarrow\; &0, &\quad \theta \rightarrow\; &0
\;.
\end{aligned}
\end{equation}
As formulated, this is essentially a "double-shooting" problem. For this homework, we make the following assumptions to simplify analysis:
\begin{enumerate}
\item More boundary conditions are known. Specifically,
\begin{equation}
\theta' =
\begin{cases}
-0.5671 &\text{if Pr} = 1 \\
-1.17 &\text{if Pr} = 10
\end{cases}
\end{equation}
\item Thus the problem is reduced to a "single-shooting" problem, with coupled equations. Good starting values for the missing BC at $\eta = 0$ are 0.6 for $\text{Pr} = 1$; and 0.41 for $\text{Pr} = 10$.
\item With $\Delta\eta = 0.02$, integrate these equations over the domain $0 \le \eta \le 10$.
\end{enumerate}

Use the fourth-order Runge-Kutta method coupled with the secant method to numerically integrate this set of equations for $\text{Pr} = \{1,10\}$. It is sufficient to set the convergence criterion for the root finder to $\epsilon = 10^{-3}$.

Plot $F$, $F'$, and $\theta$ as a function of $\eta$ for each case, and discuss the differences between the two solutions.

%%%%%%%%%%%%%%%%%%%%%%%%%%%%%%%%%%%%%%%%%%%%%%%%%
%%%%%%%%%%%%%%%%%%%%%%%%%%%%%%%%%%%%%%%%%%%%%%%%%
\section{Methodology} %%%%%%%%%%%%%%%%%%%%%%%%%%%
%%%%%%%%%%%%%%%%%%%%%%%%%%%%%%%%%%%%%%%%%%%%%%%%%
%%%%%%%%%%%%%%%%%%%%%%%%%%%%%%%%%%%%%%%%%%%%%%%%%

\subsection{Problem 1}

Note that the system \eqref{eq:prob1} can be re-written as
\begin{equation}
\begin{alignedat}{2}
\sqrt{x_2} - x_2 + \tfrac{1}{4} &= x_1 \\
f(x_2) \; = \; 8 \left( \sqrt{x_2} - x_2 + \tfrac{1}{4} \right)^2 - 8 \left( \sqrt{x_2} - x_2 + \tfrac{1}{4} \right) x_2 + 16 x_2 - 5  &= 0
\;.
\end{alignedat}
\label{eq:prob1rewrite}
\end{equation}
In this form, we apply a simple one-dimensional root finding algorithm to $x_2$, and then calculate the exact value of $x_1$.

We use the \textbf{bisection method} to determine $x_2$. First, we calculate values of $f(x)$ at the values $x = \{ x_{2_0}, x_{2_0} \pm h \}$. If the sign of $f(x)$ changes over one of these two intervals, we bisect the interval and evaluate $f(x)$ at the bisection point, recursively approaching the true value of $x_2$. We stop when our interval is less than $\epsilon$. If the sign does not change within the interval $x_{2_0} \pm h$, the user must provide a more accurate guess of $x_{2_0}$, or decrease $h$ in the case of multiple roots.

For this problem, we choose $h=0.3$ and $x_{2_0} = 0.3$, so that $x_2=0$ is still included in our initial guess, as requested in the problem statement.

\subsection{Problem 2}

To integrate the differential equation, we use the standard fourth-order Runge-Kutta (RK4) method. Boundary conditions are known and finite at $\eta = 0$, so this is where we start integration. To approximate $\eta = \infty$, it is sufficient to apply the corresponding `far-field' boundary conditions at $\eta = 10$. Since the exact form of the governing equations is known, the only unknown is $F''(\eta)$ at $\eta = 0$. We seek the appropriate value of $F''(0)$ with the secant method.

%%%%%%%%%%%%%%%%%%%%%%%%%%%%%%%%%%%%%%%%%%%%%%%%%
%%%%%%%%%%%%%%%%%%%%%%%%%%%%%%%%%%%%%%%%%%%%%%%%%
\section{Results} %%%%%%%%%%%%%%%%%%%%%%%%%%%%%%%
%%%%%%%%%%%%%%%%%%%%%%%%%%%%%%%%%%%%%%%%%%%%%%%%%
%%%%%%%%%%%%%%%%%%%%%%%%%%%%%%%%%%%%%%%%%%%%%%%%%

\subsection{Problem 1}

We find $\{ x_1, x_2 \} = \{ 0.5000000000, 0.2499999046 \}$. For curiosity's sake, convergence behavior of the solution for $x_2$ is presented in \figref{fig:prob1_convergence}.

\begin{figure}[h!]
\begin{center}
\includegraphics[width=\textwidth]{Problem1_Convergence.eps}
\\
\caption{Convergence behavior of the bisection method as it solves for $x_2$.}
\label{fig:prob1_convergence}
\end{center}
\end{figure}

\subsection{Problem 2}

%%%%%%%%%%%%%%%%%%%%%%%%%%%%%%%%%%%%%%%%%%%%%%%%%
%%%%%%%%%%%%%%%%%%%%%%%%%%%%%%%%%%%%%%%%%%%%%%%%%
\section{Discussion} %%%%%%%%%%%%%%%%%%%%%%%%%%%%
%%%%%%%%%%%%%%%%%%%%%%%%%%%%%%%%%%%%%%%%%%%%%%%%%
%%%%%%%%%%%%%%%%%%%%%%%%%%%%%%%%%%%%%%%%%%%%%%%%%

\subsection{Problem 1}

Analytical evaluation reveals that $\{ x_1, x_2 \} = \{ \tfrac{1}{2}, \tfrac{1}{4} \}$ is the solution to \eqref{eq:prob1}. In this light, the bisection method is entirely adequate in producing the correct solution, but it required 20 steps compared to the presumably fewer steps the secant method would have required.

Furthermore, in solving the $x_2$-equation in \eqref{eq:prob1rewrite} directly rather than the coupled equations in \eqref{eq:prob1}, our numerical tolerance only applies to $x_2$. For a more challenging system of equations, one would need to propagate the tolerance in $x_2$ to determine tolerance in $x_1$. Since implementing the root-finding procedure is the primary objective of this assignment, we note only that $\epsilon$ can be decreased by the user if they desire a more accurate value of $x_1$.

\subsection{Problem 2}

%%%%%%%%%%%%%%%%%%%%%%%%%%%%%%%%%%%%%%%%%%%%%%%%%
%%%%%%%%%%%%%%%%%%%%%%%%%%%%%%%%%%%%%%%%%%%%%%%%%
\section{References} %%%%%%%%%%%%%%%%%%%%%%%%%%%%
%%%%%%%%%%%%%%%%%%%%%%%%%%%%%%%%%%%%%%%%%%%%%%%%%
%%%%%%%%%%%%%%%%%%%%%%%%%%%%%%%%%%%%%%%%%%%%%%%%%

No external references were used other than the course notes for this assignment.

%%%%%%%%%%%%%%%%%%%%%%%%%%%%%%%%%%%%%%%%%%%%%%%%%
%%%%%%%%%%%%%%%%%%%%%%%%%%%%%%%%%%%%%%%%%%%%%%%%%
\section*{Appendix: MATLAB Code} %%%%%%%%%%%%%%%%
%%%%%%%%%%%%%%%%%%%%%%%%%%%%%%%%%%%%%%%%%%%%%%%%%
%%%%%%%%%%%%%%%%%%%%%%%%%%%%%%%%%%%%%%%%%%%%%%%%%

The following code listings generate all figures presented in this homework assignment.

\includecode{Problem_1.m}
\includecode{Bisect1D.m}
%\includecode{Problem_2.m}

%%
%% DOCUMENT END
%%
\end{document}
